\documentclass[14pt]{extarticle}
\usepackage[utf8]{inputenc}
\usepackage[left=2cm, right=2cm, top=1cm, bottom=1cm]{geometry}
\usepackage{amsmath}
\usepackage{amsthm}
\usepackage{amssymb}
\usepackage{mathtools}
\usepackage{pgfplots}
\pgfplotsset{compat=1.18}
\usepackage{graphicx}
\usepackage{enumitem}
\usepackage{setspace}
\usepackage{etoolbox}
\usepackage{fancyhdr}
\usepackage{anyfontsize}

% Configurar el estilo de página
\pagestyle{fancy}
\fancyhf{} % Limpiar todos los campos
\fancyfoot[C]{\thepage} % Colocar número de página en el centro
\renewcommand{\footrulewidth}{0pt} % Quitar la línea del pie de página
\renewcommand{\headrulewidth}{0pt} % Quitar la línea superior
\setlength{\footskip}{21pt} % Ajustar la distancia del número de página

% Aumentar el espacio entre columnas
\setlength{\columnsep}{6em} % Puedes ajustar el valor según necesites

% Definiciones para entornos matemáticos
\newtheorem{theorem}{Teorema}[section]
\newtheorem{proposition}[theorem]{Proposición}
\newtheorem{lemma}[theorem]{Lema}
\newtheorem{corollary}[theorem]{Corolario}
\newtheorem{definition}[theorem]{Definición}
\newtheorem{example}{Ejemplo}[section]
\newtheorem{remark}{Observación}[section]
\newtheorem{note}{Notación}[section]
\newtheorem{axiom}[theorem]{Axioma}

% Agregar después de las definiciones de newtheorem, antes de \begin{document}

% Alternativa: ajustar espacios específicos para cada tipo de entorno
\BeforeBeginEnvironment{theorem}{\vspace{15pt}}
\AfterEndEnvironment{theorem}{\vspace{15pt}}

\BeforeBeginEnvironment{definition}{\vspace{15pt}}
\AfterEndEnvironment{definition}{\vspace{15pt}}

\BeforeBeginEnvironment{example}{\vspace{15pt}}
\AfterEndEnvironment{example}{\vspace{15pt}}

\BeforeBeginEnvironment{remark}{\vspace{15pt}}
\AfterEndEnvironment{remark}{\vspace{15pt}}

\title{Acordeon Casella}
\author{Carmen Dení}
\date{\today}

\begin{document}
\fontsize{18}{22}\selectfont
% Iniciar el modo landscape
% Contenido directo sin columnas

%%%%%%%%%%%%%%%%%%%%%%%%%%%%%%%%%%%%%%%%%%%%%%%%%%%%%%%%%%%%%%%%%%%    Capitulo 1    %%%%%%%%%%%%%%%%%%%%%%%%%%%%%%%%%%%%%%%%%%%%%%%%%%%%%%%%%%%
%%%%%%%%%%%%%%%%%%%%%%%%%%%%%%%%%%%%%%%%%%%%%%%%%%%%%%%%%%%%%%%%%%%%%%%%%%%%%%%%%%%%%%%%%%%%%%%%%%%%%%%%%%%%%%%%%%%%%%%%%%%%%%%%%%%%%%%%%%%%%%%%
\section{Teoría de probabilidad}

\subsection{Teoría de conjuntos}
Lo primero que debemos hacer es identificar los posibles resultados del experimento, es decir,
el espacio muestral $S$.
\begin{definition}
    El espacio muestral $S$ es el conjunto de todos los posibles resultados de un experimento aleatorio.
\end{definition}
\begin{remark}
    El espacio muestral puede ser numerable o no numerable.
\end{remark}
\begin{definition}
    Un evento es una colección de posibles resultados de un experimento aleatorio, es decir,
     cualquier subconjunto del espacio muestral incluyendo a $S$ mismo.
     Para un evento $A$ decimos que $A$ ocurre si el resultado del experimento es un elemento de $A$.
\end{definition}

\begin{definition}
    Sean $A$ y $B$ eventos (o conjuntos).
    \begin{enumerate}
        \item $A \subseteq B \iff (x \in A \implies x \in B)$
        \item $A \cup B = \{x \in S : x \in A \text{ o } x \in B\}$
        \item $A \cap B = \{x \in S : x \in A \text{ y } x \in B\}$
        \item $A \setminus B = \{x \in S : x \in A \text{ y } x \notin B\}$
        \item $A \Delta B = (A \setminus B) \cup (B \setminus A)$
        \item $A^c = \{x \in S : x \notin A\}$
    \end{enumerate}
\end{definition}
\begin{theorem}
    Sean $A, B, C$ eventos definidos en un espacio muestral $S$. Entonces se cumplen las siguientes propiedades:
    \begin{enumerate}
        \item Conmutatividad: $A \cup B = B \cup A$ y $A \cap B = B \cap A$
        \item Asociatividad: $A \cup (B \cup C) = (A \cup B) \cup C$ y $A \cap (B \cap C) = (A \cap B) \cap C$
        \item Distributividad: $A \cap (B \cup C) = (A \cap B) \cup (A \cap C)$ y $A \cup (B \cap C) = (A \cup B) \cap (A \cup C)$
        \item Leyes de De Morgan: $(A \cup B)^c = A^c \cap B^c$ y $(A \cap B)^c = A^c \cup B^c$
        \item Leyes de De Morgan generalizadas: sea $\alpha$ una colección arbitraria de eventos, entonces:     
        \begin{align*}
            (\bigcup_{i=1}^\infty A_i)^c = \bigcap_{i=1}^\infty A_i^c \\
            (\bigcap_{i=1}^\infty A_i)^c = \bigcup_{i=1}^\infty A_i^c
        \end{align*}
    \end{enumerate}
\end{theorem}

\begin{definition}
    Decimos que $A$ y $B$ son mutuamente excluyentes si $A \cap B = \emptyset$.
    Los eventos $A_1, A_2, \ldots$ son mutuamente excluyentes si $A_i \cap A_j = \emptyset$
    para todo $i \neq j$.
\end{definition}

\begin{definition}
    Si $A_1, A_2, \ldots$ son mutuamente excluyentes, decimos que la colección es una partición de $S$ si:
    \begin{enumerate}
        \item $A_i \neq \emptyset, \forall i$
        \item $A_i \cap A_j = \emptyset, \forall i \neq j$
        \item $\bigcup_{i=1}^\infty A_i = S$
    \end{enumerate}
\end{definition}

%%%%%%%%%%%%%%%%%%%%%%%%%%%%%%%%%%%%%%%%%%%%%%%%%%%%%%%%%%%%%%%%%%%    1.2    %%%%%%%%%%%%%%%%%%%%%%%%%%%%%%%%%%%%%%%%%%%%%%%%%%%%%%%%%%%
\subsection{Teoría de probabilidad básica}
\subsubsection{Fundamentos axiomáticos}
\begin{definition}
    Dado un espacio muestral $S$, una sigma-álgebra $\mathcal{B}$ sobre $S$ es una colección de
     subconjuntos de $S$ que satisface:
    \begin{enumerate}
        \item $\emptyset \in \mathcal{B}$
        \item Si $A \in \mathcal{B}$, entonces $A^c \in \mathcal{B}$
        \item Si $A_1, A_2, \ldots \in \mathcal{B}$, entonces $\bigcup_{i=1}^\infty A_i \in \mathcal{B}$
    \end{enumerate}
\end{definition}

\begin{proposition}
    Sea $S$ un espacio muestral. Los siguientes conjuntos son sigma-álgebras sobre $S$:
    \begin{enumerate}
        \item $\mathcal{B} = \{\emptyset, S\}$
        \item $\mathcal{B} = \{s: s \subseteq S\}$
        \item Dadas $\mathcal{A}$ y $\mathcal{B}$ sigma-álgebras sobre $S$, $\mathcal{A} \cap \mathcal{B}$ es una sigma-álgebra sobre $S$.
    \end{enumerate}
\end{proposition}

\begin{definition}
    Una función $P$ definida sobre una sigma-álgebra $\mathcal{B}$ es una función de probabilidad si satisface las siguientes propiedades:
    \begin{enumerate}
        \item $P(A) \geq 0$ para todo $A \in \mathcal{B}$
        \item $P(S) = 1$
        \item Si $A_1, A_2, \ldots$ son mutuamente excluyentes, entonces:
        \[
            P(\bigcup_{i=1}^\infty A_i) = \sum_{i=1}^\infty P(A_i)
        \] 
        A esta propiedad se le conoce como aditividad numerable.
    \end{enumerate}
    A estas propiedades se les conoce como axiomas de Kolmogorov.
\end{definition}

\begin{axiom}
    Si $A$ y $B$ son mutuamente excluyentes, entonces $$P(A \cup B) = P(A) + P(B)$$.
    A esta propiedad se le conoce como aditividad finita.
\end{axiom}

\begin{proposition}
    Sea $A_1 \supset A_2 \supset \cdots \supset A_n \supset \cdots$ una sucesión de eventos anidados tales que su intersección es vacía.
    Son equivalentes los siguientes puntos:
    \begin{enumerate}
        \item
        \begin{enumerate}
            \item $P(A_i) \to 0$ cuando $i \to \infty$ 
            \item Aditividad finita
        \end{enumerate}
        \item Aditividad numerable
    \end{enumerate}
\end{proposition}

\begin{theorem}
    Sea  $S = \{s_1, s_2, \ldots, s_n\}$ un espacio muestral finito. Sea $\mathcal{B}$ una sigma algebra de $S$.
    Sean $p_1, p_2, \ldots, p_n$ números no negativos tales que $\sum_{i=1}^n p_i = 1$. Para cualquier $A \in \mathcal{B}$,
    definimos:
    \[
        P(A) = \sum_{\{ i: s_i \in A \}}^n p_i
    \]
    Entonces $P$ es una función de probabilidad sobre $\mathcal{B}$. Esto se verifica tambien para $S = \{s_1, s_2, \ldots\}$.
\end{theorem}

%%%%%%%%%%%%%%%          %%%%%%%%%%%%%%%          %%%%%%%%%%%%%%%          %%%%%%%%%%%%%%%          %%%%%%%%%%%%%%%          %%%%%%%%%%%%%%%                   
\subsubsection{Calculo probabilidades}
\begin{theorem}
    Si P es una funcion de probabilidad y $A$ cualquier conjunto en $\mathcal{B}$, entonces:
    \begin{enumerate}
        \item $P(\emptyset) = 0$
        \item $P(A) \leq 1$
        \item $P(A^c) = 1 - P(A)$
    \end{enumerate}
\end{theorem}

\begin{theorem}
    Si $P$ es una función de probabilidad y $A, B \in \mathcal{B}$, entonces:
    \begin{enumerate}
        \item $P(B \cap A^c) = P(B) - P(A \cap B)$
        \item $P(A \cup B) = P(A) + P(B) - P(A \cap B)$
        \item Si $A \subseteq B$, entonces $P(A) \leq P(B)$
        \item Si $P(A \cap B) \leq 1$, entonces $P(A \cap B) = P(A) + P(B) - 1$
    \end{enumerate}
\end{theorem}

\begin{theorem}
    Si $P$ es una función de probabilidad, entonces
    \begin{enumerate}
        \item $P(A) = \sum_{i=1}^\infty P(A \cap C_i)$ para cualquier partición $C_1, C_2, \ldots$ de $S$
        \item $P(\bigcup_{i=1}^\infty A_i) \leq \sum_{i=1}^\infty P(A_i)$ para cualquier colección de eventos $A_1, A_2, \ldots$
    \end{enumerate}
\end{theorem}


%%%%%%%%%%%%%%%          %%%%%%%%%%%%%%%          %%%%%%%%%%%%%%%          %%%%%%%%%%%%%%%          %%%%%%%%%%%%%%%          %%%%%%%%%%%%%%%          
\subsubsection{Conteo}

\begin{theorem}
    Si un trabajo consiste en k tareas sucesivas, y la tarea $i$ puede ser realizada de $n_i$ maneras, entonces el trabajo
    puede ser realizado de $n_1 \times n_2 \times \cdots \times n_k$ maneras.
\end{theorem}

\begin{remark}
    Se puede contar con remplazo o sin remplazo, tomando en cuenta el orden o no.
\end{remark}

\begin{definition}
Para un entero positivo $n$, $n!$ denota el producto de los enteros positivos desde 1 hasta $n$.
Es decir: $$n! = 1 \times 2 \times \cdots \times n$$.
Definimos $0! = 1$.
\end{definition}

\begin{definition}
    Para enteros no negativos $n$ y $r$, con $n \geq r$, definimos el simbolo $\binom{n}{r}$ como el numero de subconjuntos
    de $r$ elementos de un conjunto de $n$ elementos. Este simbolo se lee "r en n" y se calcula como:
    \[
        \binom{n}{r} = \frac{n!}{r!(n-r)!}
    \]
\end{definition}

\begin{proposition}
    Supongamos que $n$ y $r$ son enteros no negativos con $n \geq r$ y supongamos que queremos elegir $r$ objetos de un conjunto
    de $n$ objetos. Podemos hacerlo de las siguientes maneras:
    \begin{enumerate}
        \item Ordenados sin reemplazo.
        \begin{align*}  
            \frac{n!}{(n-r)!}
        \end{align*}
        \item Ordenados con reemplazo.
        \begin{align*}
            n^r
        \end{align*}
        \item No ordenados sin reemplazo.
        \begin{align*}
            \binom{n}{r}
        \end{align*}
        \item No ordenados con reemplazo.
        \begin{align*}
            \binom{n+r-1}{r}
        \end{align*}
    \end{enumerate}
\end{proposition}


%%%%%%%%%%%%%%%          %%%%%%%%%%%%%%%          %%%%%%%%%%%%%%%          %%%%%%%%%%%%%%%          %%%%%%%%%%%%%%%          %%%%%%%%%%%%%%%          
\subsubsection{Enumerar los resultados de un experimento}

\begin{definition}
    Sea $S = \{s_1, s_2, \ldots, s_N\}$ un espacio muestral finito. Decimos que todos los resultados 
    son igualmente probables si $$P(\{s_i\}) = \frac{1}{N}$$ para todo $i = 1, 2, \ldots, N$.
\end{definition}

\begin{proposition}
    Si $S$ es un espacio muestral finito con $N$ resultados igualmente probables, entonces para cualquier evento $A$,
    \[
        P(A) = \sum_{\{s_i \in A\}} P(\{s_i\}) = \sum_{\{i: s_i \in A\}}^N \frac{1}{N} = \frac{|A|}{N}
    \]
\end{proposition}

\begin{proposition}
    Sean $n$ y $r$ enteros no negativos con $n \geq r$. Pensemos que queremos saber cuantos vectores $(x_1, x_2, \ldots, x_r)$ de longitud $r$ distintos
    podemos formar tales que la suma de sus componentes es $n$. Entonces:
    \begin{enumerate}
        \item Para $x_i > 0$, entonces el numero de vectores es $\binom{n+r-1}{r}$
        \item Para $x_i \geq 0$, entonces el numero de vectores es $\binom{n+r-1}{r}$
    \end{enumerate}
\end{proposition}
    
\begin{proposition}
    Si tenemos $k$ lugares y y $m$ objetos repetidos $k_1, k_2, \ldots, k_m$ con $k_1 + k_2 + \cdots + k_m = k$, entonces
    el numero de formas de asignar los $k$ lugares a los $m$ objetos es:
    \[
        \frac{k!}{k_1! k_2! \cdots k_m!}
    \]
\end{proposition}

\begin{proposition}
    Sea $n$ un entero no negativo. Entonces:
    \begin{enumerate}
        \item $\sum_{k=0}^n \binom{n}{k}x^k = (x+1)^n$ en particular\\ $\sum_{k=0}^n \binom{n}{k} = 2^n$
        \item $\sum_{k=0}^n (-1)^k \binom{n}{k} = 0$
    \end{enumerate}
\end{proposition}

\begin{proposition}
    Sean $n$ y $r$ enteros no negativos con $n \geq r$. El número de formas de distribuir $n$ objetos entre $r$ conjuntos
    tal que cada conjunto reciba al menos un objeto está dado por:
    \[
        \sum_{k=0}^r (-1)^{k+1}\binom{r}{k}(r-k)^n
    \]
    A esta forma de conteo se le llama principio de inclusion-exclusion.
\end{proposition}

%%%%%%%%%%%%%%%%%%%%%%%%%%%%%%%%%%%%%%%%%%%%%%%%%%%%%%%%%%%%%%%%%%%    1.3    %%%%%%%%%%%%%%%%%%%%%%%%%%%%%%%%%%%%%%%%%%%%%%%%%%%%%%%%%%%
\subsection{Probabilidad condicional e independencia}
En muchos casos podemos redefinir el espacio muestral para plasmar información adicional sobre el experimento. En ese caso deberíamos poder
recalcular las probabilidades de los eventos en el nuevo espacio muestral.

\begin{definition}
    Si $A$ y $B$ son eventos en $S$con $P(B) > 0$, entonces la probabilidad condicional de $A$ dado $B$ es:
    \[
        P(A|B) = \frac{P(A \cap B)}{P(B)}
    \]
\end{definition}

\begin{proposition}
    Sea $B$ un evento con $P(B) > 0$ entonces la función $P(A|B)$ definida para todo $A \in \mathcal{B}$ es una función de probabilidad,
    es decir satisface los axiomas de Kolmogorov.
\end{proposition}

\begin{proposition}
    Sea $B$ un evento con $P(B) > 0$. Entonces para cualquier evento $A$:
    \[
        P(A|B) = P(B|A) \frac{P(A)}{P(B)}
    \]
\end{proposition}


\begin{theorem}
    Sea $A_1, A_2, \ldots$ una partición de $S$ y cualquier evento $B$. Entonces:
    \[
        P(A_i|B) = \frac{P(B|A_i)P(A_i)}{\sum_{j=1}^\infty P(B|A_j)P(A_j)}
    \]
    Para cada $i = 1, 2, \ldots$.
\end{theorem}

\begin{definition}
    Si un evento $B$ no impacta la ocurrencia de otro evento $A$, entonces:
    \[
        P(A|B) = P(A)
    \]
    Decimos que $A$ y $B$ son independientes si $$P(A \cap B) = P(A)P(B)$$.
\end{definition}

\begin{theorem}
    Si $A$ y $B$ son independientes, entonces los siguientes pares de eventos tambien son independientes:
    \begin{enumerate}
        \item $A$ y $B^c$
        \item $A^c$ y $B$
        \item $A^c$ y $B^c$
    \end{enumerate}
\end{theorem}

\begin{definition}
    Decimos que los eventos $A_1, A_2, \ldots, A_n$ son independientes si para cualquier \\ 
    subcolección $A_{i_1}, A_{i_2}, \ldots, A_{i_k}$ de $A_1, A_2, \ldots, A_n$ se cumple que:
    \[
        P(\bigcap_{j=1}^k A_{i_j}) = \prod_{j=1}^k P(A_{i_j})
    \]
\end{definition}



%%%%%%%%%%%%%%%%%%%%%%%%%%%%%%%%%%%%%%%%%%%%%%%%%%%%%%%%%%%%%%%%%%%    1.4    %%%%%%%%%%%%%%%%%%%%%%%%%%%%%%%%%%%%%%%%%%%%%%%%%%%%%%%%%%%
\subsection{Variables aleatorias}



En muchos casos la información relevante de un experimento aleatorio no está en los resultados del experimento, sino en ciertas funciones de esos resultados.

\begin{definition}
    Una variable aleatoria es una función de un espacio muestral $S$ a los números reales.

    Si X es una variable aleatoria con rango $Y = \{y_1, y_2, \ldots\}$, la función de probabilidad $P_{X}$ de X está definida de la
    siguiente manera: 
    \[ P_X(X=y_i) = P(\{s \in S : X(s) = y_i\}) \]

    Si queremos describir el valor de X para conjuntos no numerables, escribimos:
    \[ P_X(X \in A) = P(\{s \in S : X(s) \in A\}) \]
\end{definition}

\begin{note}
    Denotaremos a las variables aleatorias con letras mayúsculas y a los valores que pueden tomar con letras minúsculas.
    Por ejemplo, si $X$ es una variable aleatoria, $x$ es un valor que puede tomar $X$.
\end{note}


%%%%%%%%%%%%%%%%%%%%%%%%%%%%%%%%%%%%%%%%%%%%%%%%%%%%%%%%%%%%%%%%%%%    1.5    %%%%%%%%%%%%%%%%%%%%%%%%%%%%%%%%%%%%%%%%%%%%%%%%%%%%%%%%%%%
\subsection{Funciones de distribución}
\begin{definition}
    La función de distribución acumulada (fda) de una variable aleatoria $X$ es:
    \[
        F_X(x) = P(X \leq x), \quad \forall x \in \mathbb{R}
    \]
    
\end{definition}

\begin{example}
    Un ejemplo típico de una función de distribución acumulada es el de una distribución normal estándar:
    \begin{center}
    \begin{tikzpicture}[scale=0.8]
    \begin{axis}[
        xlabel=$x$,
        ylabel={$F_X(x)$},
        ylabel style={rotate=-90, anchor=east, at={(ticklabel cs:0.5)}, xshift=-10pt},
        xmin=-3,
        xmax=3,
        ymin=0,
        ymax=1,
        axis lines*=left,
        no marks
    ]
    \addplot[blue,thick,jump mark right] coordinates {
        (-3, 0)
        (-2, 0.2)
        (-1, 0.4)
        (0, 0.6)
        (1, 0.8)
        (2, 0.95)
        (3, 0.95)
    };
    \addplot[only marks, mark=*, mark size=2pt, mark options={fill=white}] coordinates {
        (-2, 0.2)
        (-1, 0.4)
        (0, 0.6)
        (1, 0.8)
    };
    \end{axis}
    \end{tikzpicture}
    \end{center}
\end{example}

\begin{theorem}
    Una función $F_X(x)$ es una función de distribución acumulada de una variable aleatoria $X$ si y solo si:
    \begin{enumerate}
        \item $\lim_{x \to -\infty} F_X(x) = 0$ y $\lim_{x \to \infty} F_X(x) = 1$.
        \item $F_X(x)$ es no decreciente.
        \item $F_X(x)$ es continua por la derecha, es decir, para cualquier $x_0 \in \mathbb{R}$ se cumple que:
        \[
            \lim_{x \to x_0^+} F_X(x) = F_X(x_0)
        \]
        
    \end{enumerate}
\end{theorem}

\begin{definition}
    Sea X una variable aleatoria. 
    \begin{enumerate}
        \item Decimos que $X$ es continua si $F_X(x)$ es continua.
        \item Decimos que $X$ es discreta si $F_X(x)$ es una función escalonada .
    \end{enumerate}
\end{definition}

\begin{definition}
    Sean $X$ y $Y$ variables aleatorias. Decimos que $X$ y $Y$ son identicamente distribuidas si $P(X \in A) = P(Y \in A)$
     para todo $A \in \mathcal{B}$.
\end{definition}

\begin{remark}
    La definición anterior no implica que $X = Y$.
\end{remark}

\begin{theorem}
    Las dos siguientes afirmaciones son equivalentes:
    \begin{enumerate}
        \item $X$ y $Y$ son identicamente distribuidas.
        \item $F_X(x) = F_Y(x)$ para todo $x \in \mathbb{R}$
    \end{enumerate}
\end{theorem}

%%%%%%%%%%%%%%%%%%%%%%%%%%%%%%%%%%%%%%%%%%%%%%%%%%%%%%%%%%%%%%%%%%    1.6    %%%%%%%%%%%%%%%%%%%%%%%%%%%%%%%%%%%%%%%%%%%%%%%%%%%%%%%%%%%%
\subsection{Función de densidad y función de masa de probabilidad}
\begin{definition}
    La función de masa de probabilidad de una variable aleatoria discreta $X$ es:
    \[
        f_X(x) = P(X = x)
    \]
\end{definition}

\begin{remark}
    Notemos que:
    \begin{enumerate}
        \item Para el caso discreto $P(X \leq b) = \sum_{x \leq b} f_X(x) = F_X(b)$.
        \item Para el caso continuo evaluar en un punto nos daría cero pero
         $$P(X \leq b) = \int_{-\infty}^b f_X(x) dx = F_X(b)$$ y por el teorema fundamental del cálculo si $F_X(x)$
          es diferenciable
         $$\frac{d}{dx} F_X(x) = f_X(x)$$. 
    \end{enumerate}
\end{remark}

\begin{definition}
    La función de densidad de probabilidad de una variable aleatoria continua $X$ es la función que satisface:
    \[
        F_X(x) = \int_{-\infty}^x f_X(x) dx
    \]
\end{definition}

\begin{note}
    \begin{enumerate}
        \item Para decir que la distribución de $X$ esta dada por $F_X(x)$ escribimos $X \sim F_X(x)$.
        \item Tambien podemos escribir $X \sim f_X(x)$.
        \item Si $X$ y $Y$ son variables aleatorias identicamente distribuidas, escribimos $X \sim Y$.
    \end{enumerate}
\end{note}

\begin{theorem}
    Una función $f_X(x)$ es una función de densidad de probabilidad de una variable aleatoria $X$ si y solo si:
    \begin{enumerate}
        \item $f_X(x) \geq 0, \forall x$
        \item $\int_{-\infty}^\infty f_X(x) dx = 1$ o $\sum_{x \in \mathbb{R}} f_X(x) = 1$
    \end{enumerate}
\end{theorem}

\begin{remark}
    Cualquier función $h$ no negativa en $A \in \mathcal{B}$ cuya integral o suma sobre $A$ sea finita,
     puede ser interpretada como una función de densidad de probabilidad, es decir, si:
     \[
        \int_{\{x \in A\}} h(x) dx \leq K 
     \]
      para alguna constante $K>0$, entonces $$f_X(x) = \frac{h(x)}{K}$$ es una función de densidad de probabilidad de 
      $X:A \to \mathbb{R}$.
\end{remark}






% Aquí puedes empezar a escribir tus resultados matemáticos
% Algunos entornos útiles:
% \begin{theorem} ... \end{theorem}
% \begin{proof} ... \end{proof}
% \begin{equation} ... \end{equation}
% \begin{align*} ... \end{align*}

%%%%%%%%%%%%%%%%%%%%%%%%%%%%%%%%%%%%%%%%%%%%%%%%%%%%%%%%%%%%%%%%%%%    Capitulo 2    %%%%%%%%%%%%%%%%%%%%%%%%%%%%%%%%%%%%%%%%%%%%%%%%%%%%%%%%%%%

%%%%%%%%%%%%%%%%%%%%%%%%%%%%%%%%%%%%%%%%%%%%%%%%%%%%%%%%%%%%%%%%%%%%%%%%%%%%%%%%%%%%%%%%%%%%%%%%%%%%%%%%%%%%%%%%%%%%%%%%%%%%%%
\section{Transformaciones y esperanza}
%%%%%%%%%%%%%%%%%%%%%%%%%%%%%%%%%%%%%%%%%%%%%%%%%%%%%%%%%%%%%%%%%%%    Capitulo 2   %%%%%%%%%%%%%%%%%%%%%%%%%%%%%%%%%%%%%%%%%%%%%%%%%%%%%%%%%%%
%%%%%%%%%%%%%%%%%%%%%%%%%%%%%%%%%%%%%%%%%%%%%%%%%%%%%%%%%%%%%%%%%%%%%%%%%%%%%%%%%%%%%%%%%%%%%%%%%%%%%%%%%%%%%%%%%%%%%%%%%%%%%%%%%%%%%%%%%%%%%%%%
\subsection{Funciones de distribución de una variable aleatoria}

\begin{proposition}
    Si $X$ es una variable aleatoria y $g$ es una función medible, entonces $Y = g(X)$ es una variable aleatoria.
    Como $Y$ es una función de $X$ podemos describir el comportamiento probabilístico de $Y$ en términos de $X$.
    \[
        P(Y \in A) = P(g(X) \in A) 
    \]
\end{proposition}

\begin{proposition}
    La función de probabilidad dada por 
    \begin{align*}
        P(Y\in A) &= P(g(X) \in A)\\
        &= P(\{ x \in \mathcal{X}: g(x) \in A\})\\
        &= P(X \in g^{-1}(A))
    \end{align*}        
\end{proposition}

\begin{proposition}
    Si $X$ y $Y$ son variables aleatorias continuasy $g$ es una función medible, entonces la función de distribución de $Y$ está dada por:
    \begin{align*}
        F_Y(y) &= P(Y \leq y)\\
        &= P(g(X) \leq y)\\
        &= P(\{x \in \mathcal{X}: g(x) \leq y\})\\
        &= \int_{\{x \in \mathcal{X}: g(x) \leq y\}} f_X(x) dx
    \end{align*}    
\end{proposition}

\begin{definition}
    Sea $X$ una variable aleatoria. Llamamos soporte de $X$ al conjunto:
    \[
        \text{soporte}(X) = \{x \in \mathcal{X}: f_X(x) > 0\}
    \]
\end{definition}

\begin{remark}
    Al hacer transformaciones es importante tener en mente cual es el espacio muestral de cada variable aleatoria. Notemos que  si 
    \[
        \mathcal{X} = \{x: f_X(x) > 0\}
    \]
    entonces el soporte de $Y = g(X)$ es:
    \[
        \mathcal{Y} = \{y: f_Y(y) > 0\} = \{y: y = g(x) \text{ para algún } x \in \mathcal{X}\}
    \]
\end{remark}

\begin{theorem}
    Sea $X$ con función de distribución $F_X(x)$ y $Y = g(X)$. $\mathcal{X}=\{x: f_X(x) > 0\}$ y $\mathcal{Y}=\{y: y = g(x) \text{ para algún } x \in \mathcal{X}\}$.
    Entonces:
    \begin{enumerate}
        \item Si $g$ es una función creciente en $\mathcal{X}$, entonces:
        \[
            F_Y(y) = F_X(g^{-1}(y)), \quad \text{para } y \in \mathcal{Y}
        \]
        \item Si $g$ es una función decreciente en $\mathcal{X}$, entonces:
        \[
            F_Y(y) = 1 - F_X(g^{-1}(y)), \quad \text{para } y \in \mathcal{Y}
        \]
    \end{enumerate}
\end{theorem}

\begin{corollary}
    Si $g$ es una función monótona:
    \[
        f_Y(y) = \frac{d}{dy} F_Y(y) = \begin{cases}
            f_X(g^{-1}(y))\frac{d}{dy} g^{-1}(y), & \text{para } g \text{ creciente}\\
            -f_X(g^{-1}(y))\frac{d}{dy} g^{-1}(y), & \text{para } g \text{ decreciente}
        \end{cases} 
    \]
\end{corollary}

\begin{theorem}
    Sea $X$ una variable aleatoria con función de densidad $f_X(x)$ y $Y = g(X)$ con $g$ una función monótona. Sean $\mathcal{X}=\{x: f_X(x) > 0\}$ y $\mathcal{Y}=\{y: y = g(x) \text{ para algún } x \in \mathcal{X}\}$.
    Entonces:
    \[
        f_Y(y) = \begin{cases}
            f_X(g^{-1}(y)) \left| \frac{d}{dy} g^{-1}(y) \right|, & \text{para } y \in \mathcal{Y}\\
            0, & \text{en otro caso}
        \end{cases} 
    \]
\end{theorem}

\begin{theorem}
    Sea $X$ una variable aleatoria con función de densidad $f_X(x)$ y $Y = g(X)$. Definimos a $\mathcal{X}=\{x: f_X(x) > 0\}$.
    Supongamos que existe una partición $A_1, A_2, \ldots A_k$ de $\mathcal{X}$ tal que $g$ es monótona en cada $A_i$ para $i=1,2,\ldots, k$ y tal que $P(A_0) = 0$.
    y $f_i(x)$ continua en cada $A_i$, respectivamente satisfaciendo:
    \begin{enumerate}
        \item $g_X(x) = g_{A_i}(x)$ para $x \in A_i$
        \item $g_i$ monótona en $A_i$
        \item El conjunto $\mathcal{Y}=\{y: y = g(x) \text{ para algún } x \in A_i\}$ es el mismo para todo $i$.
    \end{enumerate}
    Entonces:
    \[
        f_Y(y) = \sum_{i=0}^\infty f_{A_i}(g^{-1}(y)) \left| \frac{d}{dy} g^{-1}(y) \right|
    \]
\end{theorem}

Si tenemos una $F_X(x)$ continua y estrictamente creciente, entonces podemos definir una variable aleatoria $Y = F_X(X)$ y aplicar el teorema anterior para obtener una $Y$

\begin{definition}
    Si $F_X(x)$ es no decreciente entonces podemos definir para $0<y<1$
    $$F_X^{-1}(y) = \inf \{x: F_X(x) \geq y\}$$
\end{definition}

\begin{remark}
    $F_X^{-1}(y)$ definida como en la definición anterior es una función de distribución y coincide con la inversa de $F_X(x)$ en el caso en el que $F_X(x)$ es estrictamente creciente.
\end{remark}


\begin{theorem}[\textbf{Transformación integral de probabilidad}]
    Sea $X$ con función de distribución $F_X(x)$ continua y definimos una variable aleatoria $Y = F_X(X)$. Entonces $Y$ tiene distribución uniforme en el
     intervalo $(0,1)$, es decir:
     $$f_Y(y) = 1 \text{ o, equivalentemente }$$
     $$P(Y \leq y) = y \text{ para } 0 < y < 1$$
\end{theorem}



%%%%%%%%%%%%%%%%%%%%%%%%%%%%%%%%%%%%%%%%%%%%%%%%%%%%%%%%%%%%%%%%%%%%%%%%%%%%%%%%%%%%%%%%%%%%%%%%%%%%%%%%%%%%%%%%%%%%%%%%%%%%%%%%%%%%%%%%%%%%%%%%

\subsection{Esperanza}
La esperanza o valor esperado es simplemente un promedio ponderado por la probabilidad de los posibles valores que puede tomar una variable aleatoria.

\begin{definition}
    La esperanza de una variable aleatoria $g(X)$ donde $X$ tiene función de densidad $f_X(x)$ es:
    \[
        E(g(X)) = \begin{cases}
            \sum_{x \in \mathcal{X}} g(x) f_X(x), & \text{si } X \text{ es discreta}\\
            \int_\mathcal{X} g(x) f_X(x) dx, & \text{si } X \text{ es continua}
        \end{cases}
    \]
    En el caso de que la integral o la suma esten definidas, en caso contrario se dice que la esperanza no existe.
\end{definition}

\begin{theorem}
    Sea $X$ una variable aleatoria y $a, b, c$ constantes. Entonces para cualquier $g_1(X), g_2(X)$ cuya esperanza existe:
    \begin{enumerate}
        \item $E(ag_1(X) + bg_2(X) + c) = aE(g_1(X)) + bE(g_2(X)) + c$
        \item Si $g_1(X) \geq 0$, entonces $E(g_1(X)) \geq 0$.
        \item Si $g_1(X) \geq g_2(X)$ para todo $x$, entonces $E(g_1(X)) \geq E(g_2(X))$.
        \item Si $a \leq g_1(x) \leq b$ para todo $x$, entonces $a \leq E(g_1(X)) \leq b$.
    \end{enumerate}
\end{theorem}

\begin{definition}
    La mediana de una variable aleatoria $X$ es un número $m$ tal que $$P(X \leq m) = \frac{1}{2} \text{ y } P(X \geq m) = \frac{1}{2}$$.
    (En el caso continuo satisface:
    $$ \int_{-\infty}^m f_X(x) dx = \int_m^\infty f_X(x) dx = \frac{1}{2} $$
    ).
\end{definition}

\begin{proposition}
    Si $b=E(X)$, entonces 
    \begin{enumerate}
        \item $E(X-b) = 0$
        \item $\min E((X-b)^2) = E(X-E(X))^2$
        \item Si $X$ es una v.a. continua y $m$ la mediana de $X$ y $a$ cualquier otro número, entonces $\min_a E(|X-a|) = E(|X-m|)$.
        \item $\frac{d}{dx}E(X-a)^2 = 0 \text{ si y solo si } a = E(X)$
    \end{enumerate}
\end{proposition}




%%%%%%%%%%%%%%%%%%%%%%%%%%%%%%%%%%%%%%%%%%%%%%%%%%%%%%%%%%%%%%%%%%%    Capitulo 3    %%%%%%%%%%%%%%%%%%%%%%%%%%%%%%%%%%%%%%%%%%%%%%%%%%%%%%%%%%%
%%%%%%%%%%%%%%%%%%%%%%%%%%%%%%%%%%%%%%%%%%%%%%%%%%%%%%%%%%%%%%%%%%%%%%%%%%%%%%%%%%%%%%%%%%%%%%%%%%%%%%%%%%%%%%%%%%%%%%%%%%%%%%%%%%%%%%%%%%%%%%%%
\section{Familias comunes de distribuciones}


%%%%%%%%%%%%%%%%%%%%%%%%%%%%%%%%%%%%%%%%%%%%%%%%%%%%%%%%%%%%%%%%%%%    Capitulo 4    %%%%%%%%%%%%%%%%%%%%%%%%%%%%%%%%%%%%%%%%%%%%%%%%%%%%%%%%%%%
%%%%%%%%%%%%%%%%%%%%%%%%%%%%%%%%%%%%%%%%%%%%%%%%%%%%%%%%%%%%%%%%%%%%%%%%%%%%%%%%%%%%%%%%%%%%%%%%%%%%%%%%%%%%%%%%%%%%%%%%%%%%%%%%%%%%
\section{Variables aleatorias múltiples}

%%%%%%%%%%%%%%%%%%%%%%%%%%%%%%%%%%%%%%%%%%%%%%%%%%%%%%%%%%%%%%%%%%%    Capitulo 5    %%%%%%%%%%%%%%%%%%%%%%%%%%%%%%%%%%%%%%%%%%%%%%%%%%%%%%%%%%%
%%%%%%%%%%%%%%%%%%%%%%%%%%%%%%%%%%%%%%%%%%%%%%%%%%%%%%%%%%%%%%%%%%%%%%%%%%%%%%%%%%%%%%%%%%%%%%%%%%%%%%%%%%%%%%%%%%%%%%%%%%%%%%%%%%%%
\section{Propiedades de una muestra aleatoria}
\subsection{Propiedades básicas de una muestra aleatoria}
%%%%%%%%%%%%%%%%%%%%%%%%%%%%%%%%%%%%%%%%%%%%%%%%%%%%%%%%%%%%%%%%%%%    Capitulo 6    %%%%%%%%%%%%%%%%%%%%%%%%%%%%%%%%%%%%%%%%%%%%%%%%%%%%%%%%%%%
%%%%%%%%%%%%%%%%%%%%%%%%%%%%%%%%%%%%%%%%%%%%%%%%%%%%%%%%%%%%%%%%%%%%%%%%%%%%%%%%%%%%%%%%%%%%%%%%%%%%%%%%%%%%%%%%%%%%%%%%%%%%%%%%%%%%
\section{Principios de reducción de datos}

%%%%%%%%%%%%%%%%%%%%%%%%%%%%%%%%%%%%%%%%%%%%%%%%%%%%%%%%%%%%%%%%%%%    Capitulo 6   %%%%%%%%%%%%%%%%%%%%%%%%%%%%%%%%%%%%%%%%%%%%%%%%%%%%%%%%%%%
%%%%%%%%%%%%%%%%%%%%%%%%%%%%%%%%%%%%%%%%%%%%%%%%%%%%%%%%%%%%%%%%%%%%%%%%%%%%%%%%%%%%%%%%%%%%%%%%%%%%%%%%%%%%%%%%%%%%%%%%%%%%%%%%%%%%%%%%%%%%%%%%
\subsection{Introducción}

%%%%%%%%%%%%%%%%%%%%%%%%%%%%%%%%%%%%%%%%%%%%%%%%%%%%%%%%%%%%%%%%%%%    Capitulo 7    %%%%%%%%%%%%%%%%%%%%%%%%%%%%%%%%%%%%%%%%%%%%%%%%%%%%%%%%%%%
%%%%%%%%%%%%%%%%%%%%%%%%%%%%%%%%%%%%%%%%%%%%%%%%%%%%%%%%%%%%%%%%%%%%%%%%%%%%%%%%%%%%%%%%%%%%%%%%%%%%%%%%%%%%%%%%%%%%%%%%%%%%%%%%%%%%
\section{Estimación puntual}

%%%%%%%%%%%%%%%%%%%%%%%%%%%%%%%%%%%%%%%%%%%%%%%%%%%%%%%%%%%%%%%%%%%    Capitulo 8   %%%%%%%%%%%%%%%%%%%%%%%%%%%%%%%%%%%%%%%%%%%%%%%%%%%%%%%%%%%
%%%%%%%%%%%%%%%%%%%%%%%%%%%%%%%%%%%%%%%%%%%%%%%%%%%%%%%%%%%%%%%%%%%%%%%%%%%%%%%%%%%%%%%%%%%%%%%%%%%%%%%%%%%%%%%%%%%%%%%%%%%%%%%%%%%%
\section{Pruebas de hipótesis}
\subsection{Introducción}

\begin{definition}
    Una hipótesis es una afirmación acerca del parámetro de una distribución de probabilidad.
\end{definition}

El objetivo de una prueba de hipótesis es decidir cuál de dos hipótesis mutuamente excluyentes, $H_0$ y $H_1$, es más probable que sea cierta.

\begin{definition}
Las dos hipótesis mutuamente excluyentes en una prueba de hipótesis son:
\begin{enumerate}
    \item La hipótesis nula, $H_0$, que es una afirmación que se desea poner a prueba.
    \item La hipótesis alternativa, $H_1$, que es una afirmación que se acepta si los datos muestrales proporcionan evidencia suficiente de que $H_0$ es falsa.
\end{enumerate}
\end{definition}

\begin{remark}
    Si $\theta$ denota el parámetro de una población la forma general de una prueba de hipótesis es:
    $$H_0: \theta \in \Theta_0 \text{ vs } H_1: \theta \in \Theta_1 = \Theta_0^c$$
    donde $\Theta_0$ es subconjunto de $\Theta$ y $\Theta_1 = \Theta_0^c$ es el complemento de $\Theta_0$ en $\Theta$.
\end{remark}

\begin{definition}
    Una prueba de hipótesis es una regla de decisión que especifica:
    \begin{enumerate}
        \item Para cuales valores de la muestra aleatoria $X$ la hipótesis nula $H_0$ se acepta.
        \item Para cuales valores de la muestra aleatoria $X$ la hipótesis nula $H_0$ se rechaza y la hipótesis alternativa $H_1$ se acepta.
    \end{enumerate}
    Al subconjunto del espacio muestral para el cual $H_0$ es rechazada se le llama región crítica o de rechazo de $H_0$. Al complemento de la región crítica
    se le llama región de aceptación de $H_0$.
\end{definition}

Usualmente una prueba puede especificarse en terminos de una estadística de prueba $W(\bold{X})=W(X_1, X_2, \ldots, X_n)$ una función 
de la muestra.

\subsection{Metodos para encontrar pruebas}
Se detallarán 4 métodos.

\subsubsection{Método de la razón de verosimilitud}

Recordemos que si $X_1, X_2, \ldots, X_n$ es una muestra aleatoria de una población con función de densidad $f(x;\theta)$ entonces la función de verosimilitud
es:
$$L(\theta | x_1, x_2, \ldots, x_n) = L(\theta | \bold{x}) = f(\bold{x}| \theta) = \prod_{i=1}^n f(x_i|\theta)$$

Si definimos a $\Theta$ como el espacio muestral completo las pruebas por cociente de verosimilitud se definen como:

\begin{definition}
    El estadistico de prueba para una prueba por cociente de verosimilitud para probar\\
     $H_0: \theta \in \Theta_0$ vs $H_1: \theta \in \Theta_1 = \Theta_0^c$ es:
    $$ \lambda(\bold{x}) = \frac{\sup_{\theta \in \Theta_0} L(\theta | \bold{x})}{\sup_{\theta \in \Theta} L(\theta | \bold{x})} $$
    Un test de razón de verosimilitud es una prueba de hipótesis que tiene una región de rechazo de la forma:
    $$ \{ \bold{x}: \lambda(\bold{x}) < c \} $$
    para alguna constante $c$ que depende del nivel de significancia $\alpha$ de la prueba.
\end{definition}

Notemos que en el cociente anterior, el numerador es la probabilidad máxima de la muestra $\bold{x}$ bajo la hipótesis $H_0$ y el 
denominador es la probabilidad máxima de la muestra $\bold{x}$ para cualquier valor del parámetro.

La hipotesis nula $H_0$ se rechaza si $\lambda(\bold{x})$ es pequeño pues esto quiere decir que hay mejores valores para el parámetro $\theta$
fuera de $\Theta_0$.




%%%%%%%%%%%%%%%%%%%%%%%%%%%%%%%%%%%%%%%%%%%%%%%%%%%%%%%%%%%%%%%%%%%    Capitulo 9   %%%%%%%%%%%%%%%%%%%%%%%%%%%%%%%%%%%%%%%%%%%%%%%%%%%%%%%%%%%
%%%%%%%%%%%%%%%%%%%%%%%%%%%%%%%%%%%%%%%%%%%%%%%%%%%%%%%%%%%%%%%%%%%%%%%%%%%%%%%%%%%%%%%%%%%%%%%%%%%%%%%%%%%%%%%%%%%%%%%%%%%%%%%%%%%%
\section{Intervalos de confianza}

\begin{thebibliography}{9}
    \bibitem{Casella}
        Casella, G. and Berger, R.L. (2002), \textit{Statistical Inference}, 2nd ed.
\end{thebibliography}
\end{document}