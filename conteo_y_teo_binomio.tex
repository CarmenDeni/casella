\documentclass[14pt]{extarticle}
\usepackage[utf8]{inputenc}
\usepackage[left=2cm, right=2cm, top=1cm, bottom=1cm]{geometry}
\usepackage{amsmath}
\usepackage{amsthm}
\usepackage{amssymb}
\usepackage{mathtools}
\usepackage{pgfplots}
\pgfplotsset{compat=1.18}
\usepackage{graphicx}
\usepackage{enumitem}
\usepackage{setspace}
\usepackage{etoolbox}
\usepackage{fancyhdr}
\usepackage{anyfontsize}

% Configurar el estilo de página
\pagestyle{fancy}
\fancyhf{} % Limpiar todos los campos
\fancyfoot[C]{\thepage} % Colocar número de página en el centro
\renewcommand{\footrulewidth}{0pt} % Quitar la línea del pie de página
\renewcommand{\headrulewidth}{0pt} % Quitar la línea superior
\setlength{\footskip}{21pt} % Ajustar la distancia del número de página

% Aumentar el espacio entre columnas
\setlength{\columnsep}{6em} % Puedes ajustar el valor según necesites

% Definiciones para entornos matemáticos
\newtheorem{theorem}{Teorema}[section]
\newtheorem{proposition}[theorem]{Proposición}
\newtheorem{lemma}[theorem]{Lema}
\newtheorem{corollary}[theorem]{Corolario}
\newtheorem{definition}[theorem]{Definición}
\newtheorem{example}{Ejemplo}[section]
\newtheorem{remark}{Observación}[section]
\newtheorem{note}{Notación}[section]
\newtheorem{axiom}[theorem]{Axioma}

% Agregar después de las definiciones de newtheorem, antes de \begin{document}

% Alternativa: ajustar espacios específicos para cada tipo de entorno
\BeforeBeginEnvironment{theorem}{\vspace{15pt}}
\AfterEndEnvironment{theorem}{\vspace{15pt}}

\BeforeBeginEnvironment{definition}{\vspace{15pt}}
\AfterEndEnvironment{definition}{\vspace{15pt}}

\BeforeBeginEnvironment{example}{\vspace{15pt}}
\AfterEndEnvironment{example}{\vspace{15pt}}

\BeforeBeginEnvironment{remark}{\vspace{15pt}}
\AfterEndEnvironment{remark}{\vspace{15pt}}

\BeforeBeginEnvironment{proposition}{\vspace{15pt}}
\AfterEndEnvironment{proposition}{\vspace{15pt}}

\title{Acordeon Casella}
\author{Carmen Dení}
\date{\today}

\begin{document}
\fontsize{18}{22}\selectfont
% Iniciar el modo landscape
% Contenido directo sin columnas
\section{Conteo}
\begin{theorem}
    Si un trabajo consiste en k tareas sucesivas, y la tarea $i$ puede ser realizada
    de $n_i$ maneras, entonces el trabajo puede ser realizado de $n_1 \times n_2 \times \cdots \times n_k$ maneras.
\end{theorem}

\begin{remark}
    Se puede contar con remplazo o sin remplazo, tomando en cuenta el orden o no.
\end{remark}

\begin{definition}
Para un entero positivo $n$, $n!$ denota el producto de los enteros positivos desde 1 hasta $n$.
Es decir: $$n! = 1 \times 2 \times \cdots \times n$$.
Definimos $0! = 1$.
\end{definition}

\begin{definition}
    Para enteros no negativos $n$ y $r$, con $n \geq r$, definimos el simbolo $\binom{n}{r}$ como el numero de subconjuntos
    de $r$ elementos de un conjunto de $n$ elementos. Este simbolo se lee "r en n" y se calcula como:
    \[
        \binom{n}{r} = \frac{n!}{r!(n-r)!}
    \]
\end{definition}

\begin{proposition}
    Supongamos que $n$ y $r$ son enteros no negativos con $n \geq r$ y supongamos que queremos elegir $r$ objetos de un conjunto
    de $n$ objetos. Podemos hacerlo de las siguientes maneras:
    \begin{enumerate}
        \item Ordenados sin reemplazo.
        \begin{align*}  
            \frac{n!}{(n-r)!}
        \end{align*}
        \item Ordenados con reemplazo.
        \begin{align*}
            n^r
        \end{align*}
        \item No ordenados sin reemplazo.
        \begin{align*}
            \binom{n}{r}
        \end{align*}
        \item No ordenados con reemplazo.
        \begin{align*}
            \binom{n+r-1}{r-1}
        \end{align*}
    \end{enumerate}
\end{proposition}

\begin{definition}
    Sea $S = \{s_1, s_2, \ldots, s_N\}$ un espacio muestral finito. Decimos que todos los resultados 
    son igualmente probables si $$P(\{s_i\}) = \frac{1}{N}$$ para todo $i = 1, 2, \ldots, N$.
\end{definition}

\begin{proposition}
    Si $S$ es un espacio muestral finito con $N$ resultados igualmente probables, entonces para cualquier evento $A$,
    \[
        P(A) = \sum_{\{s_i \in A\}} P(\{s_i\}) = \sum_{\{i: s_i \in A\}}^N \frac{1}{N} = \frac{|A|}{N}
    \]
\end{proposition}


\begin{proposition}
    Sea $n$ un entero no negativo. Entonces:
    \begin{enumerate}
        \item $\sum_{k=0}^n \binom{n}{k}x^{n-k}y^k = (x+y)^n$ en particular 
        \item $\sum_{k=0}^n \binom{n}{k} = 2^n$
        \item $\sum_{k=0}^n \binom{n}{k}x^k = (x+1)^n$
        \item $\sum_{k=0}^n (-1)^k \binom{n}{k} = 0$
    \end{enumerate}
\end{proposition}


\begin{proposition}
    Sean $n$ y $r$ enteros no negativos con $n \geq r$. Pensemos que queremos saber cuantos vectores $(x_1, x_2, \ldots, x_r)$ de longitud $r$ distintos
    podemos formar tales que la suma de sus componentes es $n$. Entonces:
    \begin{enumerate}
        \item Si $x_i > 0$, entonces el numero de vectores es $\binom{n + r-1}{r-1}$
        \item Si $x_i \geq 0$, entonces el numero de vectores es $\binom{n-1}{r-1}$
    \end{enumerate}
\end{proposition}
    
\begin{proposition}
    Sean $n$ y $k$. Si tenemos un vector $(x_1, x_2, \ldots, x_n)$ de longitud $n$ 
    donde $x_i \in \{0,1\}$ entonces el numero de vectores distintos que podemos formar tal que 
    \[
        \sum_{i=1}^n x_i \geq k
    \text{    es   }
        \sum_{i=k}^n \binom{n}{i}
    \]
\end{proposition}

\begin{proposition}
    \begin{enumerate}
    \item Sean, $n, m$ y $r$ enteros no negativos con $n + m \geq r$, entonces:
    $$ \binom{n+m}{r} = \sum_{k=0}^r \binom{n}{k} \binom{m}{r-k} $$
    (Considerar dos clases y las formas de elegir $r$ elementos de $n+m$ de las cuales $k$ pertenecen a la primera clase y $r-k$ a la segunda clase)

    \item $$ \binom{2n}{n} = \sum_{k=0}^n \binom{n}{k}^2 $$
    \item $$ \binom{n}{k} k = (n-k + 1) \binom{n}{k-1} = n \binom{n-1}{k-1} $$
    (Podemos pensar en un comite en el que se elige primero el comite luego la silla de entre el comite o todo el comite que no esta en la silla y luego la silla
o la silla y luego todo el comite restante)
    \item Sean $n$ y $r$ enteros no negativos con $n \geq r$. El número de formas de distribuir $n$ objetos entre $r$ conjuntos
    tal que cada conjunto reciba al menos un objeto está dado por:
    \[
        \sum_{k=0}^r (-1)^{k+1}\binom{r}{k}(r-k)^n
    \]
    A esta forma de conteo se le llama principio de inclusion-exclusion.
    \item A la siguiente identidad le llamamos identidad combinatoria de Fermat:
    \[
        \binom{n}{k} = \sum_{i = k}^n \binom{i-1}{k-1} \quad n \geq k
    \]
    (¿Cuantos subconjuntos de tamaño $k$ tienen como elemento maximo $i$?)
    \item $$ \sum_{k=1}^n k \binom{n}{k} = n 2^{n-1} $$\\
    (¿Para comites de tamaño $k$ cuantos comites con silla hay o cuantas elecciones de silla para comietes de cualquier tamaño?)
    \item $$ \sum_{k=1}^n \binom{n}{k}k^2 = n (n+1) 2^{n-2} $$
    (¿Cuantos comites tenemos si elegimos un presidente y un secretario -posiblemente el mismo individuo-?)
    \item Para $n\geq 0$:
    \[
        \sum_{i=0}^n (-1)^{i}\binom{n}{i} = 0
    \]
    \item Para $i leq n$:$$\sum_{i=j}^n \binom{n}{j}\binom{j}{i} = \binom{n}{i} 2^n-i$$
    \item Si tenemos $k$ lugares y y $m$ objetos repetidos $k_1, k_2, \ldots, k_m$ con $k_1 + k_2 + \cdots + k_m = k$, entonces
    el numero de formas de asignar los $k$ lugares a los $m$ objetos es:
    \[
        \frac{k!}{k_1! k_2! \cdots k_m!}
    \]
    \end{enumerate}
\end{proposition}






% Aquí puedes empezar a escribir tus resultados matemáticos
% Algunos entornos útiles:
% \begin{theorem} ... \end{theorem}
% \begin{proof} ... \end{proof}
% \begin{equation} ... \end{equation}
% \begin{align*} ... \end{align*}

\end{document}