\documentclass{article}
\usepackage[utf8]{inputenc}

\title{Mi Primer Documento}
\author{Tu Nombre}
\date{\today}

\begin{document}

\maketitle

¡Hola mundo! Este es mi primer documento en \LaTeX.
Entonces, ¿cómo se hace un documento en \LaTeX?

\end{document}

% Para ver el PDF en tiempo real:
% 1. Instala la extensión "LaTeX Workshop" en VS Code
% 2. Abre el archivo settings.json (Ctrl+Shift+P -> "Preferences: Open Settings (JSON)")
% 3. Agrega estas configuraciones:
%
% {
%   "latex-workshop.latex.autoBuild.run": "onSave",
%   "latex-workshop.view.pdf.viewer": "tab", 
%   "latex-workshop.latex.outDir": "./out",
%   "latex-workshop.message.error.show": true
% }
%
% 4. Guarda el archivo settings.json
% 5. Reinicia VS Code
% 
% Ahora cada vez que guardes el archivo .tex, se compilará automáticamente
% y podrás ver el PDF actualizado en una pestaña al lado
